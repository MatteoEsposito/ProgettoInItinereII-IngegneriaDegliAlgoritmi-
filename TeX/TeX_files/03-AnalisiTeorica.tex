\chapter{Analisi delle Prestazioni: Caso Peggiore}%: Ricordati di inserire anceh il caso uno dei due vuoto}
\section{Analisi Teorica}
L'analisi delle prestazioni, nel caso peggiore, consente di fatto di stabilire quanto sia pi\'u o meno performante un dato algoritmo. L'analisi di questo algoritmo in particolare consta di diversi snipset di codice che contengono prevalentemente operazioni il cui tempo di esecuzione sia pari all'$O(m)$ per il Corollario 4 del Paper di Brandes, il resto dell'algoritmo è dominato dal ciclo esterno di Complessità pari ad un O(n).
\newline
\newline

\begin{minipage}{0.49\linewidth}
	\begin{Verbatim}[frame=topline,numbers=left,label=Codice,framesep=3mm]
C = dict((v,0) for v in V)
for s in V:
	S = []
	P = dict((w,[]) for w in V)
	g = dict((t, 0) for t in V); g[s] = 1
	d = dict((t,-1) for t in V); d[s] = 0
	Q = deque([])
	Q.append(s)
	\end{Verbatim}
\end{minipage}\hfill
\begin{minipage}{0.49\linewidth}
	\begin{Verbatim}
	L'algoritmo di brandes chiede subito di
	allocare  spazio  per  un  dizionario C 
	inizializzando tutte le sue chiavi a 0 
	ed  indicizzandole  con gli ID dei nodi 
	del grafo. 
	L'iterazione principale viene effetuata
	ponendo come nodo  "s" ( sorgente )
	uno dopo l'altro tutti i nodi del  grafo, 
	è lecito quindi asserire che ciò aggiunga 
	subito una Complessità Temporale pari 
	ad un O(n).
	L'algoritmo prosegue poi ad iniziallizare: 
	una Pila S, contenente nodi in ordine
	non crescente di distanza(s,v);
	un dizionario P di predecessori composto da 
	quei nodi che giacciono sui cammini minimi;
	un dizionario sigma ( qui "g") contenente
	il numero di cammini minimi passanti per 
	il nodo v; un dizionario delta ( qui "d" )
	contente i valori di "dependency" del nodo v
	\end{Verbatim}

\end{minipage}
\newline \newline % iniziallizzazione dell'Algoritmo e delle Struttre Dati Ausiliarie

\begin{minipage}{0.49\linewidth}
	\begin{Verbatim}[frame=topline,numbers=left,label=Codice,framesep=3mm]
 while Q:
	v = Q.popleft()
	S.append(v)
	for w in A[v]:
	
		if d[w] < 0:
			Q.append(w)
			d[w] = d[v] + 1
		
		if d[w] == d[v] + 1:
			g[w] = g[w] + g[v]
			P[w].append(v)
	
	\end{Verbatim}
\end{minipage}\hfill
\begin{minipage}{0.8\linewidth}
	\begin{Verbatim}
	L'algoritmo itera su tutti i nodi 
	presenti nella coda Q ( alll'inizio solo 
	il nodo sorgente ) quindi appende il
	nodo attualemnte sotto esame nella
	lista S. Itera quindi su tutti i nodi
	adianceti al nodo in esame estratto
	dalla coda.
	
	Si chiede quindi se un nodo non 
	fosse già stato incontrato ed in 
	tal caso lo inserisce in coda a Q
	ed aggiorna il valore di 
	dipendenza del nodo in esame.
	
	Si chiede inoltre se il nodo v sotto
	esame giaccia o meno su di un 
	cammino minimo da s quindi in caso
	affermativo lo aggiunge alla lista
	dei predecessori del nodo sorgente
	preso in esame.
	
	Il Ciclo sui nodi adiacenti ad un nodo
	impiega effettivaemnte O(m) stando
	iterando essenzialmente su ogni singolo 
	arco
	\end{Verbatim}
\end{minipage} % Calcolo dei cammini minimi e delle dipendenze
\newline 
\newpage
%\paragraph{Attenzione:} per coerenza e continuità con quanto esposto prima nella descrizione dell'algoritmo continueremo con l'ipotesi che l'albero B sia più alto dell'albero A \newline\newline


\begin{minipage}{0.49\linewidth}
	\begin{Verbatim}[frame=topline,numbers=left,label=Codice,framesep=3mm]
	
temp = dict((v, 0) for v in V)
	
while S:
  w = S.pop()
  for v in P[w]:
     temp[v] = temp[v] + (g[v]/g[w]) * (1 + temp[w])
     if w != s:
        C[w] = C[w] + temp[w]
        
	\end{Verbatim}
\end{minipage}\hfill
\begin{minipage}{0.49\linewidth}
	\begin{Verbatim}
	 Usando tutte le informazioni sui 
	 predecessori e sui cammini minimi 
	 del nodo s passiamo alla 
	 "propagazione" della "dependency" 
	 di tali nodi per poi concludere con 
	 il computo della somma di tutti i 
	 valori di dipendenza. Dalla Pila S 
	 viene restituito in ordine non-crescente
	 di distanza da v un nodo w, 
	 viene quindi calcolata l'importanza
	 di v rispetto a tutti  i nodi nell'insieme dei
	 predecessori P, fintanto che vengono 
	 considerati nodi diversi dal nodo in
	 esame s, viene incrementato il 
	 valore di "importanza" del nodo v 
	 nel grafo, quindi al termine di questo cilcio
	 viene essenzialmente calcolata la
	 "Betweennes Centrality"
	\end{Verbatim}
\end{minipage}
 \newline\newline % Calcolo della Betweennes Centrality


\paragraph{Riepilogo:}
	Il codice così analizzato ha evidenziato il fatto che l'algoritmo indipendentemente dalla tipologia di grafo e dallo scenario della funzione principale \emph{brandes(V,A))} risulta costare un tempo simil-quadratico. L'analisi teorica ci conferma che l'algoritmo asintoticamente, quindi per grandi quantit\'a di dati, è un $O(n*m) = O(|V|*|E|) \cong O(n*(n-1)) \cong O(n^2)$

% \input{./TeX_files/Python_snipset/09} % Riepilogo
